\section{Wprowadzenie}
\sectionframe

\begin{frame}
  \frametitle{Materiały}

  \begin{itemize}
    \item {
      \texttt{haskell.mariuszrozycki.pl}

      Slajdy (w tym opisana wersja z zajęć), kody źródłowe,
      dodatkowe zadania i materiały, filmy (niedługo)
      \vspace{1em}
    }
    \item {
      \texttt{github.com/mrozycki/haskell-course-pl}

      To, co wyżej, w formie surowej (tj. pliki \TeX{}owe itp.)
      \vspace{1em}
    }
    \item {
      \texttt{learnyouahaskell.com} (ang.)

      Anglojęzyczna strona z kursem programowania w Haskellu
      \vspace{1em}
    }
  \end{itemize}
\end{frame}

\begin{frame}
  \frametitle{Narzędzia}
  \begin{itemize}
    \item {
      \textbf{WinGHCi}. 
      Kompilator i interpreter języka Haskell dla systemu Windows.

      Wchodzi w skład pakietu Haskell Platform.
      \vspace{1em}
    }
    \item {
      \textbf{Notepad++}. Edytor tekstu.

      Każdy inny edytor tekstu (vim, emacs, notatnik) też się sprawdzi.
    }
  \end{itemize}
\end{frame}

\begin{frame}
  \frametitle{Programowanie funkcjonalne (1)}
  Paradygmat (trudne słowo) programowania, odmienny od programowania
  imperatywnego (kolejne trudne słowo), w którym zamiast mówić komputerowi
  \textbf{jak} ma wykonywać obliczenia, mówimy mu \textbf{co} ma policzyć.
  \vspace{1em}

  Poza tym, tak jak nazwa wskazuje, opiera się na \textit{funkcjach}.
\end{frame}

\begin{frame}[fragile]
  \frametitle{Programowanie funkcjonalne (2)}
  Na przykład w C++ (programowanie imperatywne) napisalibyśmy:

  \begin{center}
    \begin{minipage}[c]{0.9\textwidth}
      \begin{lstlisting}[language=C++]
int n; cin >> n;

int wynik = 1;
for (int i = 0; i < n; i++) {
  wynik = wynik*i;
}

cout << wynik; 
      \end{lstlisting}
    \end{minipage}
  \end{center}
  \pause
  Aby policzyć silnię.
\end{frame}

\begin{frame}[fragile]
  \frametitle{Programowanie funkcjonalne (3)}
  Natomiast w Haskellu (programowanie funkcjonalne) napiszemy:

  \begin{center}
    \begin{minipage}[c]{0.9\textwidth}
      \begin{lstlisting}[language=haskell]
silnia 0 = 1
silnia n = n*silnia (n-1)
      \end{lstlisting}
    \end{minipage}
  \end{center}

  A w ML (również programowanie funkcjonalne) napisalibyśmy:
  \begin{center}
    \begin{minipage}[c]{0.9\textwidth}
      \begin{lstlisting}[language=ML]
fun silnia 0 = 1;
  | silnia n = n*(silnia (n-1));
      \end{lstlisting}
    \end{minipage}
  \end{center}

  W tym kursie skupimy się jednak na Haskellu.

\end{frame}
