\documentclass[12pt,a4paper]{article}
\usepackage{polski}
\usepackage[utf8]{inputenc}

\newcounter{qcounter}
\newenvironment{questions}
  {\begin{list}{\textbf{Zadanie \arabic{qcounter}}}{\usecounter{qcounter}}}%
  {\end{list}}%

\title{Haskell}
\author{Grupa C}
\date{2~maja 2014}

\begin{document}

\maketitle

Na kartce papieru napisz definicje funkcji opisanych poniżej. Nie musisz
przejmować się typami, za konflikty typów nie będą odejmowane punkty.

\begin{enumerate}
  \item {
    \texttt{mi2km d} -- przelicza podany w milach dystans $d$ na kilometry.
    Możesz przyjąć, że 1 mila to 1.6 kilometra.
  }

  \item {
    \texttt{delta a b c} -- liczy wartość wyznacznika równania kwadratowego
    postaci $ax^2+bx+c=0$, gdzie $a \neq 0$.
  }

  \item {
    \texttt{do\_zera n} -- zwraca wartość liczby różniącej się o $1$ od
    $n$, ale znajdującej się bliżej zera. W przypadku zera wynik może być
    dowolny. 
    
    Przykłady: \texttt{do\_zera (-3)} zwraca $-2$, 
    a \texttt{do\_zera 5} zwraca 4.
  }

  \item {
    \texttt{potega n k} -- liczy wartość $n^k$. Uwaga! Nie używaj wbudowanego
    w Haskella operatora \texttt{\^}.
  }
\end{enumerate}


\end{document}
