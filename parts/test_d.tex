\documentclass[12pt,a4paper]{article}
\usepackage{polski}
\usepackage[utf8]{inputenc}

\newcounter{qcounter}
\newenvironment{questions}
  {\begin{list}{\textbf{Zadanie \arabic{qcounter}}}{\usecounter{qcounter}}}%
  {\end{list}}%

\title{Haskell}
\author{Grupa D}
\date{2~maja 2014}

\begin{document}

\maketitle

Na kartce papieru napisz definicje funkcji opisanych poniżej. Nie musisz
przejmować się typami, za konflikty typów nie będą odejmowane punkty.

\begin{enumerate}
  \item {
    \texttt{m2ft d} -- przelicza podany w metrach dystans $d$ na stopy.
    Możesz przyjąć, że 1 stopa to 30,5 centymetra.
  }

  \item {
    \texttt{heron a b c} -- liczy pole trójkąta o bokach długości $a$, $b$ i $c$.
    Możesz użyć wbudowanej funkcji \texttt{sqrt}.
  }

  \item {
    \texttt{trojmian\_kwadratowy a b c} -- liczy liczbę rozwiązań równania 
    $ax^2 + bx + c = 0$.
  }

  \item {
    \texttt{liebniz k} -- liczy $k$-te przybliżenie liczby $\pi$ z szeregu 
    Liebniza. 
    
    Szereg Liebniza wygląda następująco:
    $$\sum_{n=0}^{\infty} \frac{(-1)^n}{2n+1}$$ lub bardziej po ludzku:
    $$1-\frac13 + \frac15 - \frac17 + \cdots$$
  }
\end{enumerate}


\end{document}
