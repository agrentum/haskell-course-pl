\section{Listy}
\sectionframe

\begin{frame}
  \frametitle{Podstawowa notacja list}
  Co zwróci zapytanie \texttt{[1..10]}?
  \vspace{1em}
  \pause

  Listę elementów od 1 do 10:

  \texttt{[1,2,3,4,5,6,7,8,9,10]}
  \vspace{2em}
  \pause

  Możemy też napisać \texttt{[1,3..10]}. Co otrzymamy?
  \vspace{1em}
  \pause

  \texttt{[1,3,5,7,9]}
  \vspace{2em}
  \pause

  Możemy również po prostu zapisać listę jako oddzielone przecinkami
  pojedyncze elementy, tak jak widoczne jest to w wynikach poprzednich
  dwóch zapytań.
  
\end{frame}

\begin{frame}
  \frametitle{Operacje na listach}
  Konkatenacja (łączenie) list
  
  \texttt{[2,4,6] ++ [5,3,1]}
  \vspace{1em}
  \pause

  Dołączenie nowego elementu z przodu listy

  \texttt{2:[4,3]}
  \vspace{1em}
  \pause

  Znajdowanie długości listy

  \texttt{length [5,6,8,4,2,5,1,7,9]}
  \vspace{1em}
  \pause

  Sumowanie listy

  \texttt{sum [1..10]}
  \vspace{1em}
  \pause

  Znajdowanie maksimum listy (minimum analogicznie)

  \texttt{minimum [5,6,8,4,2,5,1,7,9]}
\end{frame}

\begin{frame}
  \frametitle{Listy w funkcjach}
  Funkcje mogą zwracać listy jako wynik:

  \texttt{przedzial a b = [a..b]}
  \vspace{1em}
  \pause

  Ale mogą być także przekazywane jako argumenty:

  \texttt{srednia\_listy l = (sum l)/(length l)}
\end{frame}

\begin{frame}[fragile]
  \frametitle{Zadania}
  \begin{block}{Zadanie 5}
    Napisz funkcję \texttt{suma\_przedzialu a b}, która zwróci sumę liczb
    naturalnych od $a$ do $b$ włącznie.
  \end{block}
  \vspace{1em}
  \pause

  \begin{block}{Rozwiązanie}
    \begin{lstlisting}[language=Haskell]
suma_przedzialu a b = sum [a..b]
    \end{lstlisting}
  \end{block}
\end{frame}

\begin{frame}[fragile]
  \frametitle{Zadania}
  \begin{block}{Zadanie 6}
    Napisz funkcję \texttt{suma\_list l1 l2}, która zwróci sumaryczną
    długość list \texttt{l1} i \texttt{l2}.
  \end{block}
  \vspace{1em}
  \pause

  \begin{block}{Rozwiązanie}
    \begin{lstlisting}[language=Haskell]
suma_list l1 l2 = sum (l1 ++ l2);
    \end{lstlisting}

    Albo:

    \begin{lstlisting}[language=Haskell]
suma_list l1 l2 = sum l1 + sum l2;
    \end{lstlisting}
  \end{block}
\end{frame}
