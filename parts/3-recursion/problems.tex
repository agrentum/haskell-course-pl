\section{Podstawy rekurencji}
\begin{questions}
  \item {
    Napisz funkcję \texttt{silnia n}, która policzy wartość silni z $n$.
  }
  \item {
    Napisz funkcję \texttt{potega n k}, która policzy wartość $n^k$. Nie
    używaj wbudowanych w Haskella funkcji i operatorów potęgowania.
  }
  \item {
    Napisz funkcję \texttt{suma\_kwadratow n}, która policzy sumę kwadratów
    liczb od 1 do $n$ włącznie.
  }
  \item {
    Napisz funkcję \texttt{suma\_poteg n k}, która policzy sumę $k$-tych potęg
    liczb od 1 do $n$ włącznie. Możesz użyć własnej funkcji potęgującej lub
    wbudowanych mechanizmów Haskella, aby wykonać potęgowanie.
  }
  \item {
    Napisz funkcję \texttt{liebniz k}, która policzy $k$-te przybliżenie liczby
    $\pi$ z szeregu Liebniza. Szereg Liebniza wygląda następująco:
    $$\sum_{n=0}^{\infty} \frac{(-1)^n}{2n+1}$$ lub bardziej po ludzku:
    $$1-\frac13 + \frac15 - \frac17 + \cdots$$.
  }
\end{questions}
