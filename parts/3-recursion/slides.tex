\section{Podstawy rekurencji}
\sectionframe

\begin{frame}[fragile]
  \frametitle{Funkcja rekurencyjna}
  Funkcja rekurencyjna to taka, która w swojej definicji odwołuje się do
  siebie samej.
  \vspace{1em}
  \pause

  \begin{block}{Przykład}
    Silnia zdefiniowana jest jako:
    \begin{eqnarray}
      \nonumber 0! &=& 1\\
      \nonumber n! &=& n\cdot(n-1)!\text{, dla }n\neq 0
    \end{eqnarray}
  \end{block}
  \pause

  Co w Haskellu można zapisać jako
  \begin{lstlisting}[language=Haskell]
silnia 0 = 1
silnia n = n*silnia (n-1)
  \end{lstlisting}
\end{frame}

\begin{frame}[fragile]
  \frametitle{Potęgowanie rekurencyjne}
  Potęgę o wykładniku naturalnym również można zdefiniować rekurencyjnie:

  \begin{eqnarray}
    \nonumber a^0 &=& 1 \\
    \nonumber a^n &=& a\cdot a^{n-1}\text{, dla }n \neq 0
  \end{eqnarray}
  \pause

  W jaki sposób zapisać to w Haskellu?
  \pause
  \begin{lstlisting}[language=Haskell]
potega a 0 = 1
potega a n = a*potega a (n-1)
  \end{lstlisting}
\end{frame}

\begin{frame}[fragile]
  \frametitle{Zadania}
  \begin{block}{Zadanie 5}
    Napisz funkcję \texttt{suma\_kwadratow n}, która policzy sumę kwadratów
    liczb od 1 do $n$ włącznie.
  \end{block}
  \vspace{1em}
  \pause

  \begin{block}{Rozwiązanie}
    \begin{lstlisting}[language=Haskell]
suma_kwadratow 1 = 1
suma_kwadratow n = n^2 + suma_kwadratow (n-1)
    \end{lstlisting}
  \end{block}
\end{frame}
