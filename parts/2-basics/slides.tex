\section{Podstawy Haskella}
\sectionframe

\begin{frame}[fragile]
  \frametitle{Zapytania}
  GHCi jest interpreterem, zatem musimy wydawać mu \textit{zapytania}, aby
  otrzymać interesujący nas wynik.

  \vfill
  \begin{center}
    \begin{minipage}{0.4\textwidth}
      \Large
      \begin{verbatim}
Prelude> 2+2*2
6
it :: Integer
      \end{verbatim}
    \end{minipage}
  \end{center}
  \vfill
  \pause

  Ile wynosi $2+2^{22}$?
\end{frame}

\begin{frame}[fragile]
  \frametitle{Definiowanie funkcji}
  
  Możemy w zewnętrznym pliku zdefiniować własne funkcje, które potem
  będziemy wykorzystywać w zapytaniach.
  \vspace{1em}
  \pause

  Stwórzmy plik \texttt{funkcje.hs}, w którym napiszemy:
  \begin{center}
    \Large{\texttt{kwadrat x = x*x}}
  \end{center}
  \vspace{1em}
  \pause

  Po wczytaniu go do GHCi możemy wykonać zapytanie:
  \begin{center}
    \Large{\texttt{Main> kwadrat 23}}
  \end{center}
  \vspace{1em}
  \pause

  Jaki otrzymamy wynik?

\end{frame}

\begin{frame}[fragile]
  \frametitle{Zadania}
    \begin{block}{Zadanie 1}
    Napisz funkcję \texttt{potega\_2 n}, która policzy wartość $n$-tej
    potęgi 2. Na przykład \texttt{potega\_2 16} zwróci 65 536.
  \end{block}
  \vspace{1em}
  \pause

  \begin{block}{Rozwiązanie}
    \begin{lstlisting}[language=Haskell]
potega_2 n = 2^n
    \end{lstlisting}
  \end{block}
  \pause

  \begin{block}{Zadanie 2}
    Napisz funkcję \texttt{suma n}, która policzy wartość sumy liczb od 1 do $n$.
  \end{block}
  \vspace{1em}
  \pause

  \begin{block}{Rozwiązanie}
    \begin{lstlisting}[language=Haskell]
suma n = n*(n+1)/2
    \end{lstlisting}
  \end{block}

\end{frame}

\begin{frame}[fragile]
  \frametitle{Funkcje wielu argumentów}
  Funkcje w Haskellu mogą przyjmować więcej niż jedną wartość.
  \vspace{1em}

  \begin{lstlisting}[language=Haskell]
suma_kwadratow x y = x^2 + y^2
  \end{lstlisting}
  \vspace{1em}
  \pause

  Jaki wynik zwróci zapytanie \texttt{suma\_kwadratow 3 4}?
  \vspace{1em}
  \pause

  Jaki wynik zwróci zapytanie \texttt{suma\_kwadratow 1.2 0.5}?

\end{frame}

\begin{frame}[fragile]
  \frametitle{Zadania}
  \begin{block}{Zadanie 3}
    Napisz funkcję \texttt{srednia a b}, która policzy średnią arytmetyczną
    liczb $a$ i $b$.
  \end{block}
  \vspace{1em}
  \pause

  \begin{block}{Rozwiązanie}
    \begin{lstlisting}[language=Haskell]
srednia a b = (a+b)/2
    \end{lstlisting}
  \end{block}
\end{frame}

\begin{frame}[fragile]
  \frametitle{Zadania}
  \begin{block}{Zadanie 4}
    Napisz funkcję \texttt{jednomian a x n}, która policzy wartość jednomianu
    $ax^n$.
  \end{block}
  \vspace{1em}
  \pause

  \begin{block}{Rozwiązanie}
    \begin{lstlisting}[language=Haskell]
jednomian a x n = a*x^n
    \end{lstlisting}
  \end{block}
\end{frame}
