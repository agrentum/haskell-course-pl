\section{Podstawy Haskella}
\begin{questions}
  \item {
    Ile wynosi $2+2^{22}$?
  }
  \item {
    Napisz funkcję \texttt{kwadrat x}, która obliczy kwadrat danej liczby $x$.
  }
  \item {
    Napisz funkcję \texttt{potega\_2 n}, która policzy wartość $n$-tej
    potęgi 2. Na przykład \texttt{potega\_2 16} zwróci 65 536.
  }
  \item {
    Napisz funkcję \texttt{suma n}, która policzy wartość sumy liczb od 1 do $n$.
  }
  \item {
    Napisz funkcję \texttt{srednia a b}, która policzy średnią arytmetyczną
    liczb $a$ i $b$.
  }
  \item {
    Napisz funkcję \texttt{jednomian a x n}, która policzy wartość jednomianu
    $ax^n$.
  }
  \item {
    Napisz funkcję \texttt{newton n k}, która policzy wartość dwumianu Newtona
    $n \choose k$. Możesz użyć funkcji \texttt{silnia} ze slajdów.
  }
  \item {
    Napisz funkcję \texttt{przeciwprostokatna a b}, która policzy długość
    przeciwprostokątnej w trójkącie prostokątnym o długościach przyprostokątnych
    $a$ i $b$. Do obliczenia pierwiastka możesz użyć wbudowanej funkcji
    \texttt{sqrt}.
  }
  \item {
    Napisz funkcję \texttt{srednia\_geometryczna a b}, która policzy wartość
    średniej geometrycznej liczb $a$ i $b$. Dla przypomnienia: wynosi ona
    $\sqrt{ab}$.
  }
  \item {
    Napisz funkcję \texttt{wierzcholek a b c}, która policzy wartość funkcji
    kwadratowej $ax^2 + bx + c$ w wierzchołku.
  }
  \item {
    Napisz funkcję \texttt{heron a b c}, która policzy pole trójkąta o bokach
    $a$, $b$ i $c$.
  }
  \item {
    Napisz funkcję \texttt{pole a}, która policzy pole trójkąta równobocznego
    o boku o długości $a$.
  }
\end{questions}
