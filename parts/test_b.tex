\documentclass[12pt,a4paper]{article}
\usepackage{polski}
\usepackage[utf8]{inputenc}

\newcounter{qcounter}
\newenvironment{questions}
  {\begin{list}{\textbf{Zadanie \arabic{qcounter}}}{\usecounter{qcounter}}}%
  {\end{list}}%

\title{Haskell}
\author{Grupa B}
\date{11~kwietnia 2014}

\begin{document}

\maketitle

Na kartce papieru napisz definicje funkcji opisanych poniżej. Nie musisz
przejmować się typami, za konflikty typów nie będą odejmowane punkty.

\begin{enumerate}
  \item {
    \texttt{kwadrat x} -- liczy wartość $x^2$.
  }

  \item {
    \texttt{srednia\_geometryczna a b} -- liczy wartość średniej geometrycznej 
    liczb $a$ i $b$. Dla przypomnienia: wynosi ona $\sqrt{ab}$. Do obliczenia 
    pierwiastka możesz użyć funkcji \texttt{sqrt}.
  }

  \item {
    \texttt{minimum a b} -- wybiera mniejszą z wartości $a$ i $b$.
  }

  \item {
    \texttt{silnia n} -- liczy silnię z $n$.
  }
\end{enumerate}


\end{document}
