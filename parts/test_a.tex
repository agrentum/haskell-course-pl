\documentclass[12pt,a4paper]{article}
\usepackage{polski}
\usepackage[utf8]{inputenc}

\newcounter{qcounter}
\newenvironment{questions}
  {\begin{list}{\textbf{Zadanie \arabic{qcounter}}}{\usecounter{qcounter}}}%
  {\end{list}}%

\title{Haskell}
\author{Grupa A}
\date{11~kwietnia 2014}

\begin{document}

\maketitle

Na kartce papieru napisz definicje funkcji opisanych poniżej. Nie musisz
przejmować się typami, za konflikty typów nie będą odejmowane punkty.

\begin{enumerate}
  \item {
    \texttt{potega\_2 n} -- liczy wartość $n$-tej potęgi 2. 
    Możesz użyć wbudowanego w Haskella potęgowania.
  }

  \item {
    \texttt{srednia a b} -- liczy średnią arytmetyczną liczb $a$ i $b$.
  }

  \item {
    \texttt{wartosc\_bezwzgledna x} -- liczy wartość modułu z $x$.
  }

  \item {
    \texttt{suma\_kwadratow n} -- liczy sumę kwadratów liczb od 1 do $n$ 
    włącznie.
  }
\end{enumerate}


\end{document}
